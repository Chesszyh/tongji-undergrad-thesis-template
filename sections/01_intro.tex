\section{介绍}\label{sec:introduction}

在本节(\cref{sec:introduction})中,我们将讨论文档中常用的各种标题级别和字体样式。在文章中,标题是非常重要的组成部分之一,它可以帮助读者更好地理解文章的结构和内容。

\subsection{二级标题}

二级标题是一种较为重要的标题级别,一般用于表示文章中的主要章节或主题。通常,它们会在上面添加分割线或加粗等效果,以突出其重要性。

\subsubsection{三级标题}

相对于二级标题而言,三级标题是更加具体的标题级别,通常用于表示二级标题下的具体内容描述。它们的长度通常比二级标题短,与二级标题之间应有一定的间距。

\paragraph{段落标题}

段落标题是文章中比正文稍微具有一些重要性和突出性的内容,通常用加粗或斜体等方式来区别于正文。

\subparagraph{子段落标题}

子段落标题是相对于段落标题更加细节化的内容,用于突出一段文字中的重点内容。通常采用斜体或加粗的方式表示。在一些正式的文献中,子段落标题的使用较少。

\subsection{测试列表}

在下面的段落中,我们将展示如何使用不同的列表类型来表示不同的内容。在文档中,列表是一种非常常见的内容表示方式,它可以帮助读者更好地理解文章的结构和内容。以下内容将结合同济大学的一些相关信息进行展示。

\subsubsection{无序列表}

无序列表是一种不带编号的列表类型,通常用于表示一组相关的内容。在无序列表中,每个列表项通常由一个特殊的符号或图标来表示。

\begin{itemize}
    \item 同济大学概述:同济大学是一所以工科著称的综合性大学,具有深厚的历史和优秀的教学资源,是中国著名的高等学府之一,享有盛誉。
    \begin{itemize}
        \item 创建于1907年:同济大学始建于1907年,是中国历史最悠久的大学之一,拥有悠久的历史和丰富的文化底蕴;
        \item 位于上海市:校园位于中国上海市,这里是中国经济和文化的重要中心,为学生提供了广阔的发展空间,是学习和生活的理想之地,也是中国高等教育的重要基地;
        \item 以工科著称:同济大学在工程学科方面享有盛誉,是中国顶尖的工科学府之一,培养了大批优秀的工程技术人才,为国家的工程建设和科技创新做出了重要贡献。
    \end{itemize}
    \item 校园文化:同济大学注重校园文化建设,营造了丰富多彩的校园生活。这里的文化活动多样,包括学生社团、艺术节、体育赛事等,为学生提供了展示才华和发展兴趣的平台。
    \item 学术成就:同济大学在多个领域取得了显著的学术成就,培养了大批优秀人才。学校在科学研究、技术创新和社会服务方面都有突出的表现,为国家和社会的发展做出了重要贡献。
\end{itemize}

\subsubsection{有序列表}

有序列表是一种带编号的列表类型,通常用于表示一组按顺序排列的内容。在有序列表中,每个列表项通常由一个数字或字母来表示。

需要注意的是,根据同济大学提供的理工科毕业设计(论文)撰写规范,有序列表的第一级编号应使用全角圆括号内的数字,如“(1)”、“(2)”等;第二级编号应使用圆圈内的阿拉伯数字,如“\Circled{1}”、“\Circled{2}”等。此外,有序列表的第二级应为行内列表,即不应另起一行,而应与上一级列表项在同一行内;我们在写 \LaTeX{} 代码时,应该使用 \texttt{enumerate*} 环境来实现有序列表的第二级。

\begin{enumerate}
    \item 同济大学历史:同济大学是中国著名的高等学府之一,拥有悠久的历史和丰富的文化底蕴。以下是同济大学的重要历史节点:
    \begin{enumerate*}
        \item 1907年:创建,最初为德文医学堂;
        \item 1927年:成为国立大学,更名为同济大学;
        \item 1952年:调整学科设置,成为以工科为主的综合性大学。
    \end{enumerate*}
    \item 学术体系:同济大学在学术方面具有很高的声誉,涵盖多个重要学科领域。主要学科包括:
    \begin{enumerate*}
        \item 工程学,包括土木工程、建筑工程、机械工程等;
        \item 建筑学,包括建筑设计、城市规划等;
        \item 医学,包括临床医学、药学等。
    \end{enumerate*}
    \item 国际合作:同济大学积极参与国际交流与合作,拓展全球视野,提升国际影响力。主要的合作项目包括:
    \begin{enumerate*}
        \item 德国交流项目,包括学生交换、教师访问等;
        \item 全球合作伙伴,包括国际知名大学、科研机构等;
        \item 国际学术会议,包括学术研讨会、国际学术交流等。
    \end{enumerate*}
\end{enumerate}

\subsection{字体}

在下面的段落中,我们使用了不同的字体来表示不同的文字信息。下面是各段落所使用的字体和对应的命令:

\begin{itemize}
    \item {\songti 宋体}:使用命令 \texttt{\textbackslash songti}。
    \item {\heiti 黑体}:使用命令 \texttt{\textbackslash heiti}。
    \item {\fangsong 仿宋}:使用命令 \texttt{\textbackslash fangsong}。
    \item {\kaishu 楷书}:使用命令 \texttt{\textbackslash kaishu}。
\end{itemize}

{\songti

1900年前后,由埃里希·宝隆创办的“同济医院”正式挂牌。医院的医师大都是“德医公会”成员。他们白天忙于经营自己的诊所,只有傍晚到医院看门诊、动手术。埃里希·宝隆医生看到医院里的医疗力量不足,计划在院内设立一所德文医学堂,招收中国学生,以培养施诊医生。这个计划得到德国驻沪总领事以及德国政府高等教育司的支持。1906年,他们设立了一个支持医学堂开办的基金会,得到了德国“促进德国与外国思想交流的科佩尔基金会”的协助,筹集到一批医科书刊及新式的外科手术电动器械等物品。}

{\heiti 1907年6月医学堂开学前,德国驻沪总领事克纳佩在上海不仅号召德国商人捐款,而且要求德国洋行向中国商人募捐。同时,费舍尔还要求中国官方的资助和支持,克纳佩利用在中德两国募来的捐款,成立了“为中国人办的德国医学堂基金会”。当时规定,捐款金额较多者可成为医学堂董事会董事。医学堂建立时定名为德文医学堂,并成立了董事会负责学校的管理。董事会由18人组成,主要成员有:三个德医公会元老:宝隆、福沙伯(第二任校长)、福尔克尔;三名德国商人:莱姆克、米歇劳和赖纳;两名中国绅商:朱葆三(沪军都督府财政部长及上海商务会会长,大买办)、虞洽卿(荷兰银行买办);总领事馆的副领事弗赖海尔·冯·吕特等。埃里希·宝隆医生被正式推选为董事会总监督(董事长)兼学堂首任总理(校长),负责学堂的管理。医学堂的校址设在同济医院对面的白克路(今凤阳路415号上海长征医院内)。1907年10月1日德文医学堂举行了开学典礼。}

{\ifcsname fangsong\endcsname\fangsong\else[无 \cs{fangsong} 字体。]\fi
1923年3月17日北洋政府教育部下达第108号训令,批准同济工科“改为大学”。学校随即召开董事会议,将学校定名为“同济大学”。1923年3月26日,学校以“同济大学董事会”名义呈文北洋政府教育部,称“经校董会议定名称为同济大学”。1923年4月24日,北洋政府教育部下达第634号“指令”,称“该校名称拟改为同济大学,应予照准备案”。1924年5月20日北洋政府教育部下达第120号训令,批准同济医科为大学。从此以后,5月20日定为校庆日。}

{\ifcsname kaishu\endcsname\kaishu\else[无 \cs{kaishu} 字体。]\fi
抗日战争胜利后,1946年,国立同济大学分批迁回上海。由于缺少校舍,学校分散教学,成为斜跨上海市区、分散十多处的“大学校”。其中学校办公室和医学院位于善钟路100弄10号(今常熟路),附属医院分别为白克路上的中美医院(原宝隆医院,今凤阳路)和同孚路82号(今石门一路)的原德国医院,理学院在平昌街日本第七国民学校内(今国顺路上海电视大学),文法学院在四川北路(今复兴初级中学),新生院位于江湾新市区的市图书馆(今黑山路),高级工业职业学校则在江湾魏德迈路(今邯郸路),附属中学位于市博物馆(今长海医院飞机楼)。工学院位于其美路的原日本中学(今杨浦区四平路1239号),1949年后逐步发展成同济大学的主校区。}

\subsubsection{测试生僻字}\label{sec:uncommon}

本小节对生僻字\footnote{此处的生僻字指:GBK编码中有,但GB2312编码中没有的字。}的显示进行测试。

丂丄丅丆丏丒丗丟丠両丣並丩丮丯丱丳丵丷丼乀乁乂乄乆乊乑乕乗乚乛乢乣乤乥乧乨乪乫乬乭乮乯乲乴乵乶乷乸乹乺乻乼乽乿亀亁亃亄亅亇亊亐亖亗亙亜亝亣亪亯亰亱亴亶亷亸亹亼亽亾仈仌仏仐仒仚仛仜仠仢仦仧仩仭仮仯仱仴仸仹仺仼仾伀伂伃伄伅伆伇伈伋伌伒伓伔伕伖伜伝伡伣伨伩伬伭伮伱伳伵伷伹伻伾伿佀佁佂佄佅佇佈佉佊佋佌佒佔佖佡佢佦佨佪佫佭佮佱佲併佷佸佹佺佽侀侁侂侅侇侊侌侎侐侒侓侕侘侙侚侜侞侟価侢侤侫侭侰侱侲侳侴侶侷侸侹侺侻侼侽侾俀俁係俆俇俈俉俋俌俍俒俓俔俕俖俙俛俢俤俥俧俫俬俰俲俴俵俶俷俹俻俼俽俿倀倁倂倃倄倅倇倈倊倎倐倓倕倖倗倛倝倞倠倢倣値倧倯倰倱倲倳倴倵倶倷倸倹倻倽倿偀偁偂偄偅偆偊偋偍偐偑偒偓偔偖偗偘偙偛偝偞偟偠偡偢偣偤偦偧偨偩偪偫偭偮偯偰偱偲偳偸偹偺偼偽傁傂傃傄傆傇傉傊傋傌傎傏傐傑傒。